\documentclass[journal]{IEEEtran}
\usepackage{amsmath, amsfonts}
\usepackage{graphicx}
\usepackage{cite}


\begin{document}

\title{A Hyper-Heuristic System for Automated Meta-Heuristic Selection in Single-Objective Optimization}

\author{
\IEEEauthorblockN{Your Name}
\IEEEauthorblockA{
Department of XXXX \\
University Name \\
Email: your@email.com}
}
% The paper headers
\markboth{IEEE	, ~Vol.~X, No.~X, ~XXXX}%
{Shell \MakeLowercase{\textit{et al.}}: Bare Demo of IEEEtran.cls for
  Journals}

\maketitle

\begin{abstract}
    This paper proposes a hyper-heuristic system for automating the selection of meta-heuristic algorithms in single-objective optimization problems. We employ Differential Evolution (DE) as a hyper-heuristic to explore optimal algorithm combinations from a predefined pool, including PSO, GWO, HHO, GA, SA, and TS. 
    
    %
   Experimental results show that the proposed hyper-heuristic consistently outperforms individual meta-heuristics in convergence rate and final fitness. On CEC2020 and CEC2022, it further demonstrates strong adaptability by dynamically adjusting algorithm combinations across optimization phases. In conclusion, this research offers a direction for scalable and intelligent optimization, laying the groundwork for future learning-based hyper-heuristic systems.

\end{abstract}

\begin{IEEEkeywords}
    hyper-heuristic, meta-heuristic algorithm, automated optimization
\end{IEEEkeywords}

\section{Introduction}
    With the rapid advancement of optimization technology and the increasing complexity of 
    real-world problems, there is a growing demand for adaptive and intelligent optimization strategies. 
    Meta-heuristic algorithms have become increasingly popular for solving complex optimization problems 
    due to their flexibility and generality. However, their performance often relies heavily on 
    problem-specific tuning and prior knowledge, and may vary significantly across different problem types. 
    This raises the challenge of algorithm selection, where the most suitable algorithm 
    (or combination of algorithms) must be chosen based on the problem’s characteristics.
    
    To address this, hyper-heuristic systems have emerged as a promising solution. 
    These systems aim to automate the process of selecting and configuring meta-heuristics, 
    reducing human intervention and enhancing adaptability. 
    
    In this paper, we present a hyper-heuristic system aimed at automating the selection 
    of meta-heuristic algorithms for solving single-objective optimization problems. 
    The core idea involves using a Differential Evolution (DE) algorithm as a hyper-heuristic 
    to identify the optimal combination of meta-heuristics from a pre-defined algorithm pool, 
    which includes PSO, DE, GWO, HHO, GA, SA, and TS. The system takes labeled function descriptors 
    (e.g., function ID, dimension, and number of terms) as pseudo input and outputs an optimized list 
    of algorithm configurations.
    
    Benchmark tests are conducted on a series of single-objective functions, 
    including CEC benchmark suites from 2019 to 2023. Several evaluation metrics are considered, 
    such as objective fitness value, convergence speed, stability, rate, and consistency. 
    Additional criteria include explainable and consistent decision-making.

    Experimental results showed that the proposed hyper-heuristic system consistently outperformed 
    individual meta-heuristic algorithms in terms of convergence rate and final fitness values 
    across most benchmark functions. Particularly, on CEC2020 and CEC2022 benchmark sets, 
    the DE-based hyper-heuristic demonstrated superior adaptability by dynamically adjusting the 
    combination of algorithms during different phases of optimization. 
    The incorporation of algorithm release-time control further improved performance by 
    leveraging the strengths of specific meta-heuristics at appropriate stages. 
    Additionally, the decision-making process exhibited promising levels of consistency, 
    and preliminary analysis indicated the potential for explainable behavior patterns 
    when certain function characteristics were present.

\section{Related Works}
    Numerous studies have focused on developing adaptive or ensemble-based meta-heuristic frameworks. 
    Previous research has explored algorithm portfolios, offline and online selection strategies, 
    as well as machine learning-based meta-algorithm selection.

    Hyper-heuristics, originally proposed for combinatorial optimization, have gradually been extended to continuous domains. 
    Recent approaches include reinforcement learning-based selection, rule-based decision systems, and genetic programming. 
    However, few works explore the potential of combining these ideas with dynamic control mechanisms 
    and modern learning models like transformers.

\section{Research Methods}
    The proposed system is composed of the following components:

\subsection{Problem Formulation}
    We focus on single-objective optimization using benchmark functions such as the CEC 2019–2023 test suites. 
    Each problem is represented by its function label, dimensionality, and number of terms.

\subsection{Hyper-Heuristic Algorithm}
    We adopt Differential Evolution (DE) as the primary hyper-heuristic controller. 
    Its role is to evolve a population of meta-heuristic configurations, 
    selecting the most suitable algorithms from a predefined pool.

\subsection{Meta-Heuristic Pool}
    The algorithm pool includes: PSO, DE, GWO, HHO, GA, SA, and TS. 
    Each algorithm can be activated at specific stages during optimization, 
    allowing controlled release and greater flexibility.

\subsection{Evaluation Metrics}
    Optimization performance is evaluated based on fitness value, convergence speed, 
    convergence stability, and consistency. Qualitative analysis is also conducted on decision explainability.

\subsection{Implementation}
    The framework is implemented using Python with support from libraries such as \texttt{opfunu}, \texttt{matplotlib}, and \texttt{tqdm}.

\section{Conclusion}
    This paper introduced a DE-based hyper-heuristic system for adaptive selection 
    of meta-heuristic algorithms in single-objective optimization. 
    Experimental results demonstrate that the proposed approach outperforms individual meta-heuristics 
    in terms of convergence and adaptability.
    Future work will investigate integration with transformer-based models 
    to enhance decision-making quality and enable large-scale automated optimization.

\section*{References}
\bibliographystyle{IEEEtran}
\begin{thebibliography}{99}

\bibitem{hyper}

\bibitem{cec2020}
 
\bibitem{de}
    R. Storn and K. Price, “Differential evolution – a simple and efficient heuristic for global optimization over continuous spaces,” *Journal of Global Optimization*, vol. 11, no. 4, pp. 341–359, 1997.

% Add more as needed

\end{thebibliography}

\end{document}
